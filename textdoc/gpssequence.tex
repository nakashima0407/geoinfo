\documentclass[11pt]{jreport}
\usepackage{latexsym}
\usepackage{mathrsfs}
\usepackage{amssymb}
%\usepackage{url}
%\usepackage{lscape}
\usepackage{graphics}
\usepackage{theorem}

%% for apple LaserWriter Series %%
%% 
\setlength{\topmargin}{-0.5in}
\setlength{\textwidth}{5.6in}
\setlength{\textheight}{8.8in}
\setlength{\oddsidemargin}{0.35in}
\setlength{\evensidemargin}{0in}

\usepackage{theorem}
\renewcommand{\baselinestretch}{1.25}
\setlength{\parskip}{0.25ex}
\renewcommand{\arraystretch}{0.85}
\begin{document}


\subsection*{定義}
位置点は整数である時刻 $t$, $xy$ 平面上の座標値 $x, y$ の組 $(t, x, y) \in \mathbb{Z}^+$ である. 
位置点の時刻に関して真に昇順の列(時系列) $q = ((t_1, x_1, y_1), \ldots,(t_n,x_n,y_n))$ をログ log とよぶ.
軌跡 $q$ の位置点および各点とその次の点の間の線分からなる列
\[
\tilde{q} = (r_1, (r_1, r_2), r_2, (r_2, r_3), r_3, \ldots, r_{n-1}, (r_{n-1}, r_n), r_n)
\]
を $q$ の軌跡 trajectory という.

\begin{defn}[位置点列マッチング]
ある正の値 $\varepsilon \in \mathbb{R}^+$ について,
二つの位置点列 $q = (s_1, \ldots, s_n)$ と $r = (t_1, \ldots, t_p)$ の軌跡の $\varepsilon$共通部分列とは,
 $\tilde{q}$ と $\tilde{r}$ の点と点または点と線分の距離が $\varepsilon$ 以内であるものを同値とみなした最長共通部分列 longest common super sequence である.
\end{defn}

\end{document}
