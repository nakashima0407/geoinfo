\documentclass[11pt]{jreport}
\usepackage{latexsym}
\usepackage{mathrsfs}
\usepackage{amssymb}
%\usepackage{url}
%\usepackage{lscape}
\usepackage{graphics}
\usepackage{theorem}

%% for apple LaserWriter Series %%
%% 
\setlength{\topmargin}{-0.5in}
\setlength{\textwidth}{5.6in}
\setlength{\textheight}{8.8in}
\setlength{\oddsidemargin}{0.35in}
\setlength{\evensidemargin}{0in}

\usepackage{theorem}
\renewcommand{\baselinestretch}{1.25}
\setlength{\parskip}{0.25ex}
\renewcommand{\arraystretch}{0.85}
\begin{document}


\subsection*{定義}
位置点は整数である時刻 $t$, $xy$ 平面上の座標値 $x, y$ の組 $(t, x, y) \in \mathbb{Z}^+$ である. 
位置点の時刻に関して真に昇順の列(時系列) $q = ((t_1, x_1, y_1), \ldots,(t_n,x_n,y_n))$ を軌跡 trajectory とよぶ.

ある整数列(あるいは有限アルファベット上の文字列)$q = (q_1, q_2, \ldots, q_n)$ と列 $s = (s_1, \ldots, s_m)$ (ただし $m \leq n$)について,真に昇順である添え字の列 $i(1) < i(2) < \cdots < i(m)$ で $q_{i(1)} = s_1, q_{i(2)} = s_2, \ldots, q_{i(m)} = s_m$ を満たすものがあるとき,
$s$ は $q$ の部分列 subsequence であるという.

文字列の連続した一部である部分文字列 substring と異なり,部分列は列中の要素の間が(添え字の数字が)飛んでいてもよい.

\begin{example}
$q = (134, 135, 136, 135, 134, 132, 137)$ のとき,$s = (134, 135, 132, 137)$ は $q$ の部分列.
\end{example}

\begin{defn}[最長共通部分列問題 longest common sub-sequence problem]
整数(あるいは有限アルファベット中の文字)の列の組 $q, r$ が与えられたとき,
$q$ の部分列かつ $r$ の部分列となる列で,最も長いもの $s$ を求める問題.
\end{defn}

\begin{defn}[2つの位置点列の最長共通部分列 (I)]
ある正の値 $\varepsilon \in \mathbb{R}^+$ について,
二つの位置点列 $q = (s_1, \ldots, s_n)$ と $r = (t_1, \ldots, t_p)$ の間の $\varepsilon$共通部分列とは,
 $q$ と $r$ の点と点の距離が $\varepsilon$ 以内の対を点が等しいとみなした最長共通部分列 longest common super sequence である.
\end{defn}

\begin{defn}[2つの位置点列の最長共通部分列 (II)]
軌跡 $q$ の位置点および位置点と次の点の間の線分からなる列
\[
\tilde{q} = (r_1, (r_1, r_2), r_2, (r_2, r_3), r_3, \ldots, r_{n-1}, (r_{n-1}, r_n), r_n)
\]
を $q$ の経路 path という.
ある正の値 $\varepsilon \in \mathbb{R}^+$ について,
二つの位置点列 $q = (s_1, \ldots, s_n)$ と $r = (t_1, \ldots, t_p)$ それぞれの経路 $\tilde{q}, \tilde{r}$ の間の $\varepsilon$共通部分列とは,
 $\tilde{q}$ と $\tilde{r}$ の (1) 点と点の距離が $\varepsilon$ 以内,または (2) 点と線分の距離が $\varepsilon$ 以内である対を等しいとみなした最長共通部分列 longest common super sequence である.
\end{defn}

\begin{example}
例をつくってみよう.
\end{example}

\end{document}
